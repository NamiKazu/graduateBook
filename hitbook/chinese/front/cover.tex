% !Mode:: "TeX:UTF-8"

\hitsetup{
  %******************************
  % 注意:
  %   1. 配置里面不要出现空行
  %   2. 不需要的配置信息可以删除
  %******************************
  %
  %=====
  % 秘级
  %=====
  statesecrets={公开},
  natclassifiedindex={TP311},
  intclassifiedindex={004.02},
  %
  %=========
  % 中文信息
  %=========
  % ctitleone={局部多孔质气体静压},%本科生封面使用
  % ctitletwo={轴承关键技术的研究},%本科生封面使用
  ctitlecover={面向质量评估的代码变更影响分析及其可视化研究},%放在封面中使用,自由断行
  ctitle={面向质量评估的代码变更影响分析及其可视化研究},%放在原创性声明中使用
  % csubtitle={一条副标题}, %一般情况没有,可以注释掉
  cxueke={工学},
  csubject={软件工程},
  caffil={计算学部},
  cauthor={李美娜},
  csupervisor={苏小红教授},
  % cassosupervisor={蒋远}, % 副指导老师
  % ccosupervisor={某某某教授}, % 联合指导老师
  % 如果是深圳本科毕业论文,需要取消注释下一行,并将内容改为“规范”中要求的封面第一页最下方的日期
  szshortcdate={2022年6月},
  % 日期自动使用当前时间,若需指定按如下方式修改:
  %cdate={超新星纪元},
  cstudentid={9527},
  cstudenttype={学术学位论文}, %非全日制教育申请学位者
  cnumber={no9527}, %编号
  cpositionname={哈铁西站}, %博士后站名称
  cfinishdate={20XX年X月---20XX年X月}, %到站日期
  csubmitdate={20XX年X月}, %出站日期
  cstartdate={3050年9月10日}, %到站日期
  cenddate={3090年10月10日}, %出站日期
  %(同等学力人员)、(工程硕士)、(工商管理硕士)、
  %(高级管理人员工商管理硕士)、(公共管理硕士)、(中职教师)、(高校教师)等
  %
  %
  %=========
  % 英文信息
  %=========
  etitle={Research on Code Change Impact Analysis and Visualization for Quality Assessment},
  % esubtitle={This is the sub title},
  exueke={Engineering},
  esubject={Software Engineering},
  eaffil={\emultiline[t]{Faculty of Computer}},
  eauthor={Li Meina},
  esupervisor={Prof. Su Xiaohong},
  % eassosupervisor={XXX},
  % 日期自动生成,若需指定按如下方式修改:
  % edate={December, 2017},
  estudenttype={Master of Art},
  %
  % 关键词用“英文逗号”分割
  ckeywords={代码质量, 检索增强生成,变更影响分析, 代码度量,代码结构可视化},
  ekeywords={code review, code quality assessment, change impact analysis, Code Metrics},
}

\begin{cabstract}

在软件系统的维护过程中,代码变更不可避免,即使是微小的改动,也可能引发连锁反应,对现有代码质量和系统整体架构产生深远影响。尤其是在大型软件系统中,随着开发团队的扩展和系统复杂度的提升,长期维护可能导致代码结构逐步退化,模块化设计逐渐被复杂的耦合结构取代,从而显著增加系统的维护和管理难度。通过变更影响分析,可在变更发生前识别其潜在影响范围,有效协助开发人员实现一致性变更与安全变更,减少变更带来的风险。此外,结合变更影响分析结果,还能够进一步揭示软件代码结构中的质量问题。

本文以 C/C++ 项目为研究对象,围绕代码变更影响分析方法展开研究,并将变更影响分析结果与质量分析结合起来,通过代码审查图的形式展示。该方法不仅能清晰地揭示代码变更可能造成的影响范围与程度,还能帮助开发人员从宏观层面掌握软件代码结构和质量,从而有效降低系统维护成本和风险,确保软件系统的稳定运行。

首先,本文实现了基于依赖关系、克隆关系和历史共现关系的变更影响分析方法,在分析其局限性的基础上,提出了基于代码预训练模型的变更影响分析方法。实验结果表明,所提方法在多个指标上优于其他方法,尤其在对逻辑型变更影响关系的检测上,取得了显著的提升。

其次,针对基于代码预训练模型方法在实际应用中在计算效率和依赖信息丢失方面的缺陷,本文进一步提出了基于依赖链和检索增强生成的变更影响分析方法。通过依赖链弥补了间接依赖信息的缺失,通过离线构建知识库和在线检索的方式,结合大语言模型在代码语义理解上的优越性能,有效提高了计算效率和检测性能。实验结果表明,该方法进一步提高了变更影响的检测性能,尤其是在间接依赖型变更影响关系上的性能得到了显著提升。

最后,本文提出代码审查图的概念,以代码中的关键元素作为节点,元素间关系作为边,将变更影响分析结果结合代码质量和软件代码结构直观地展示给用户,帮助开发者和审查人员更直观地理解整个项目的质量情况和代码结构,优化项目的持续维护过程,提高代码维护过程的安全性和效率。此外,还生成了详细的代码质量分析报告。实验结果表明,代码审查图能够帮助开发人员快速聚焦于特定开发模块,有助于其发现不良的代码模式。同时,代码质量检测报告以清单形式展示了项目中各项质量度量,为开发人员提供了清晰的项目质量状况概览。

\end{cabstract}

\begin{eabstract}
  As the scale and complexity of software systems continue to increase, the traditional method of relying on experienced experts for manual code reviews has become inadequate to handle the increasingly complex code review demands. As a result, the importance of automated code quality assessment methods has become more prominent. In large software systems, as development teams expand and system complexity increases, the code structure often deteriorates over time during maintenance, with modular design being gradually replaced by complex code structures. This leads to greater difficulty in maintaining and managing the system. Furthermore, code changes are almost inevitable during the maintenance of software systems, and these changes can have a profound impact on the quality of the existing code and the overall system structure, thus exacerbating the difficulty of system maintenance.

  This paper focuses on code quality assessment, conducting in-depth research on change impact analysis methods, and combines quality assessment metrics to present the quality evaluation results of C/C++ projects in the form of code review graphs. This approach allows developers to better understand the software architecture and code quality from a macro perspective.
  
  First, this paper calculates software quality metrics based on intermediate code representations, extracting the abstract syntax tree of software projects using Clang, and further extracting method summary tables and global variable information tables from the abstract syntax tree. These intermediate code representations capture dependencies such as code calls, and based on these representations, this paper extracts relevant code quality metrics and information from the perspectives of cohesion, coupling, code complexity, and code defects. These metrics provide developers with deep insights into code quality and offer optimization directions for subsequent code optimization and maintenance decisions.
  
  Second, this paper further explores change impact analysis methods to help developers better understand the potential interdependencies that may arise during the maintenance of software projects. Change impact analysis methods can predict the possible effects of code changes, helping developers make more reasonable design and optimization decisions, thereby reducing the risks brought by changes. The paper first implements the traditional change impact analysis method based on dependency closure, and then proposes three new change impact analysis methods based on code cloning, data mining, and deep learning techniques. Experimental results show that these three new methods outperform the traditional dependency closure approach and effectively address its shortcomings.
  
  Finally, to help developers and reviewers gain a comprehensive understanding of the software project's architecture from a macro perspective, this paper presents the results of code quality analysis in the form of code review graphs. Additionally, detailed code quality inspection reports are generated. The study shows that code review graphs can help developers quickly focus on specific development modules, enabling them to understand the overall code architecture without needing to grasp excessive code context. Meanwhile, the code quality inspection report presents project quality indicators in a checklist format, providing developers with a clear overview of the project's quality status.
\end{eabstract}
