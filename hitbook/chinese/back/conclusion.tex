% !Mode:: "TeX:UTF-8" 
\begin{conclusions}

随着软件系统规模和复杂性的不断增加,确保代码质量已成为一个日益重要的挑战。传统的依赖经验丰富的专家进行人工代码审查的方法,已经难以满足现代软件开发,特别是大型系统中的需求。随着系统的复杂性增加,代码结构往往会逐渐退化,导致系统的维护和管理变得更加困难。因此,自动化、高效的代码质量评估方法显得尤为迫切。本文面向代码质量评估,对代码变更影响分析方法进行研究,取得了如下成果:

(1)本文通过使用Clang提取了三种代码中间表示形式——抽象语法树、方法摘要表和全局变量信息表,并进一步计算了12种质量度量。这些度量包括基于内聚度缺乏度和连通度内聚度的6项度量,4种耦合关系度量以及2个代码复杂度度量。同时,结合静态分析工具对代码缺陷进行检测。研究结果表明,这些度量指标具有较高的准确性,能够有效帮助开发者全面了解软件的质量状况。

(2)本文实现了基于传统依赖闭包的代码变更影响分析方法,并提出了三种新的变更影响分析方法。基于代码克隆的方法通过检测软件项目中的代码克隆情况,反映出代码变更影响关系。基于数据挖掘的方法通过挖掘代码变更历史,检测历史中频繁共同更改的方法,提示用户变更影响,避免变更不完全。基于深度学习的方法利用数据挖掘中提取的数据作为数据集进行训练,能够在没有变更历史的情况下,预测代码变更的影响关系。实验结果表明,这三种方法均在不同的角度优于传统方法,并且能有效弥补传统方法只能挖掘基于依赖关系的变更影响关系的不足。

(3)本文提出了代码审查图的方式,将代码质量分析结果展示给开发者,方便开发者或审查者从宏观的角度了解软件项目架构,聚焦特定模块,减少上下文阅读。本文提出了基于大语言模型的方法模块预测方法,帮助用户识别模块错误划分的方法。除此之外还生成详细的代码质量检测报告,为开发人员提供了清晰的项目质量状况概览。

但本文的研究方法依然存在一些不足,在未来工作中,可考虑进一步在以下方面进行研究:

(1)对于大规模软件项目,代码审查图可能过于复杂,信息难以辨识。因此,可以考虑分层展示细节,首先按模块级、方法级等不同层级构建图,用户可以通过点击模块节点查看更为详细的子图,从而实现信息的逐层递进展示。

(2)基于大语言模型的方法模块划分可进一步尝试融合聚类和大模型预测的方法,提高准确性。

\end{conclusions}
