
% 各种引用
\section{引用}[Cite]

\sindex[china]{du!段誉}引文标注遵照GB/T7714-2005,采用顺序编码制。正文中引用文献的标示应置于所引内容最后一个字的右上角,所引文献编号用阿拉伯数字置于方括号“[ ]”中,用小4号字体的上角标。要求:

(1)引用单篇文献时,如“二次铣削\cite{cnproceed}”。

(2)同一处引用多篇文献时,各篇文献的序号在方括号内全部列出,各序号间用“,”,如
遇连续序号,可标注讫序号。如,…形成了多种数学模型\cite{cnarticle,cnproceed}…
注意此处添加\cs{inlinecite} inlinecite 中文空格\inlinecite{cnarticle,cnproceed},可以在cfg文件中修改空格类型。

(3)多次引用同一文献时,在文献序号的“[ ]”后标注引文页码。如,…间质细胞CAMP含量
测定\cite[100-197]{cnarticle}…。…含量测定方法规定
\cite[92]{cnarticle}…。



% 定义定理
\section{定理和定义等}[Theorem]
\begin{theorem}[\cite{cnproceed}]
宇宙大爆炸是一种爆炸。
\end{theorem}

% 表格
\begin{table}[htbp]
\bicaption[table1]{}{符合研究生院绘图规范的表格}{Table$\!$}{Table in agreement of the standard from graduate school}
\vspace{0.5em}\centering\wuhao
\begin{tabular}{ccccc}
\toprule
$D$(in) & $P_u$(lbs) & $u_u$(in) & $\beta$ & $G_f$(psi.in)\\
\midrule
    5 & 269.8 & 0.000674 & 1.79 & 0.04089\\
10 & 421.0 & 0.001035 & 3.59 & 0.04089\\
20 & 640.2 & 0.001565 & 7.18 & 0.04089\\
\bottomrule
\end{tabular}
\end{table}

% 长表格
\vspace{-1.5bp}
\ltfontsize{\wuhao[1.667]}
\wuhao[1.667]\begin{longtable}{ccc}%
\longbionenumcaption{}{{\wuhao 中国省级行政单位一览}\label{table2}}{Table$\!$}{}{{\wuhao Overview of the provincial administrative unit of China}}{-0.5em}{3.15bp}\\
%\caption{\wuhao 中国省级行政单位一览}\label{table2}\\
\toprule 名称 & 简称 & 省会或首府\\ \midrule
\endfirsthead
\multicolumn{3}{r}{表~\thetable(续表)}\vspace{0.5em}\\
\toprule 名称 & 简称 & 省会或首府\\ \midrule
\endhead
\midrule[0.5pt]
\endfoot
\bottomrule
\endlastfoot
北京市 & 京 & 北京\\
天津市 & 津 & 天津\\
河北省 & 冀 & 石家庄市\\
山西省 & 晋 & 太原市\\
内蒙古自治区 & 蒙 & 呼和浩特市\\
辽宁省 & 辽 & 沈阳市\\
吉林省 & 吉 & 长春市\\
黑龙江省 & 黑 & 哈尔滨市\\
上海市 & 沪/申 & 上海\\
江苏省 & 苏 & 南京市\\
浙江省 & 浙 & 杭州市\\
安徽省 & 皖 & 合肥市\\
福建省 & 闽 & 福州市\\
江西省 & 赣 & 南昌市\\
山东省 & 鲁 & 济南市\\
河南省 & 豫 & 郑州市\\
湖北省 & 鄂 & 武汉市\\
湖南省 & 湘 & 长沙市\\
广东省 & 粤 & 广州市\\
广西壮族自治区 & 桂 & 南宁市\\
海南省 & 琼 & 海口市\\
重庆市 & 渝 & 重庆\\
四川省 & 川/蜀 & 成都市\\
贵州省 & 黔/贵 & 贵阳市\\
云南省 & 云/滇 & 昆明市\\
西藏自治区 & 藏 & 拉萨市\\
陕西省 & 陕/秦 & 西安市\\
甘肃省 & 甘/陇 & 兰州市\\
青海省 & 青 & 西宁市\\
宁夏回族自治区 & 宁 & 银川市\\
新疆维吾尔自治区 & 新 & 乌鲁木齐市\\
香港特别行政区 & 港 & 香港\\
澳门特别行政区 & 澳 & 澳门\\
台湾省 & 台 & 台北市\\
\end{longtable}\normalsize
\vspace{-1em}




% 图片
\begin{figure}[h]
\centering
\includegraphics[width = 0.4\textwidth]{golfer}
\caption{打高尔夫球的人,硕士论文要求只用汉语}
\end{figure}


\begin{figure}[htbp]
\centering
\begin{minipage}{0.4\textwidth}
\centering
\includegraphics[width=\textwidth]{golfer}
\bicaption[golfer2]{}{打高尔夫球的人}{Fig.$\!$}{The person playing golf}
\end{minipage}
\centering
\begin{minipage}{0.4\textwidth}
\centering
\includegraphics[width=\textwidth]{golfer}
\bicaption[golfer3]{}{打高尔夫球的人。注意,这里默认居中}{Fig.$\!$}{The person playing golf. Please note that, it is vertically center aligned by default.}
\end{minipage}
\end{figure}


\begin{figure}[htbp]
\centering
\begin{minipage}[t]{0.4\textwidth}
\centering
\includegraphics[width=\textwidth]{golfer}
\bicaption[golfer5]{}{打高尔夫球的人}{Fig.$\!$}{The person playing golf}
\end{minipage}
\centering
\begin{minipage}[t]{0.4\textwidth}
\centering
\includegraphics[width=\textwidth]{golfer}
\bicaption[golfer8]{}{打高尔夫球的人。注意,此图是顶部对齐}{Fig.$\!$}{The person playing golf. Please note that, it is vertically top aligned.}
\end{minipage}
\end{figure}

\begin{figure}[htbp]
\centering
\begin{minipage}[t]{0.4\textwidth}
\centering
\includegraphics[width=\textwidth,height=\textwidth]{golfer}
\bicaption[golfer9]{}{打高尔夫球的人。注意,此图对齐方式是图片底部对齐}{Fig.$\!$}{The person playing golf. Please note that, it is vertically bottom aligned for figure.}
\end{minipage}
\centering
\begin{minipage}[t]{0.4\textwidth}
\centering
\includegraphics[width=\textwidth]{golfer}
\bicaption[golfer6]{}{打高尔夫球的人}{Fig.$\!$}{The person playing golf}
\end{minipage}
\end{figure}

\begin{figure}[!h]
\setlength{\subfigcapskip}{-1bp}
\centering
\begin{minipage}{\textwidth}
\centering
\subfigure{\label{golfer41}}\addtocounter{subfigure}{-2}
\subfigure[The person playing golf]{\subfigure[打高尔夫球的人~1]{\includegraphics[width=0.4\textwidth]{golfer}}}
\hspace{2em}
\subfigure{\label{golfer42}}\addtocounter{subfigure}{-2}
\subfigure[The person playing golf]{\subfigure[打高尔夫球的人~2]{\includegraphics[width=0.4\textwidth]{golfer}}}
\end{minipage}
\centering
\begin{minipage}{\textwidth}
\centering
\subfigure{\label{golfer43}}\addtocounter{subfigure}{-2}
\subfigure[The person playing golf]{\subfigure[打高尔夫球的人~3]{\includegraphics[width=0.4\textwidth]{golfer}}}
\hspace{2em}
\subfigure{\label{golfer44}}\addtocounter{subfigure}{-2}
\subfigure[The person playing golf. Here, 'hang indent' and 'center last line' are not stipulated in the regulation.]{\subfigure[打高尔夫球的人~4。注意,规范中没有明确规定要悬挂缩进、最后一行居中。]{\includegraphics[width=0.4\textwidth]{golfer}}}
\end{minipage}
\vspace{0.2em}
\bicaption[golfer4]{}{打高尔夫球的人}{Fig.$\!$}{The person playing gol}
\end{figure}

\begin{figure}[t]
\centering
\begin{minipage}{.7\linewidth}
    \setlength{\subfigcapskip}{-1bp}
    \centering
    \begin{minipage}{\textwidth}
    \centering
    \subfigure{\label{golfer45}}\addtocounter{subfigure}{-2}
    \subfigure[The person playing golf]{\subfigure[打高尔夫球的人~1]{\includegraphics[width=0.4\textwidth]{golfer}}}
    \hspace{4em}
    \subfigure{\label{golfer46}}\addtocounter{subfigure}{-2}
    \subfigure[The person playing golf]{\subfigure[打高尔夫球的人~2]{\includegraphics[width=0.4\textwidth]{golfer}}}
    \end{minipage}
    \vskip 0.2em
\wuhao 注意:这里是中文图注添加位置(我工要求,图注在图题之上)。
    \vspace{0.2em}
\bicaption[golfer47]{}{打高尔夫球的人。注意,此处我工有另外一处要求,子图图题可以位于主图题之下。但由于没有明确说明位于下方具体是什么格式,所以这里不给出举例。}{Fig.$\!$}{The person playing golf. Please note that, although it is appropriate to put subfigures' captions under this caption as stipulated in regulation, but its format is not clearly stated.}
\end{minipage}
\end{figure}

\begin{figure}[t]
\centering
\begin{tikzpicture}
    \node[anchor=south west,inner sep=0] (image) at (0,0) {\includegraphics[width=0.3\textwidth]{golfer}};
    \begin{scope}[x={(image.south east)},y={(image.north west)}]
        \node at (0.3,0.5) {a)};
        \node at (0.8,0.2) {b)};
    \end{scope}
\end{tikzpicture}
\bicaption[golfer0]{}{打高尔夫球球的人(博士论文双语题注)}{Fig.$\!$}{The person playing golf (Doctoral thesis)}
\vskip -0.4em
\hspace{2em}
\begin{minipage}[t]{0.3\textwidth}
\wuhao \setlist[description]{font=\normalfont}
    \begin{description}
        \item[(a)]子图图题
    \end{description}
\end{minipage}
\hspace{2em}
\begin{minipage}[t]{0.3\textwidth}
\wuhao \setlist[description]{font=\normalfont}
    \begin{description}
        \item[(b)]子图图题
        \item[(b)]Subfigure caption
    \end{description}
\end{minipage}
\end{figure}



\clearpage

如果不想让图片浮动到下一章节,那么在此处使用\cs{clearpage}命令。




\subsection{列宽可调表格的绘制方法}[Methods of drawing tables with adjustable-width columns]
论文中能用到列宽可调表格的情况共有两种,一种是当插入的表格某一单元格内容过长以至
于一行放不下的情况,另一种是当对公式中首次出现的物理量符号进行注释的情况,这两种
情况都需要调用~tabularx~宏包。下面将分别对这两种情况下可调表格的绘制方法进行阐述
。
\subsubsection{表格内某单元格内容过长的情况}[The condition when the contents in
some cells of tables are too long]
首先给出这种情况下的一个例子如表~\ref{table3}~所示。
\begin{table}[htbp]
  \centering
\bicaption[table3]{}{最小的三个正整数的英文表示法}{Table$\!$}{The English construction of the smallest three positive integral numbers}\vspace{0.5em}\wuhao
\begin{tabularx}{0.7\textwidth}{llX}
\toprule
Value & Name & Alternate names, and names for sets of the given size\\
\midrule
1 & One & ace, single, singleton, unary, unit, unity\\
2 & Two & binary, brace, couple, couplet, distich, deuce, double, doubleton, duad, duality, duet, duo, dyad, pair, snake eyes, span, twain, twosome, yoke\\
3 & Three & deuce-ace, leash, set, tercet, ternary, ternion, terzetto, threesome, tierce, trey, triad, trine, trinity, trio, triplet, troika, hat-trick\\
\bottomrule
\end{tabularx}
\end{table}
tabularx环境共有两个必选参数:第1个参数用来确定表格的总宽度,第2个参数用来确定每
列格式,其中标为X的项表示该列的宽度可调,其宽度值由表格总宽度确定。标为X的列一般
选为单元格内容过长而无法置于一行的列,这样使得该列内容能够根据表格总宽度自动分行
。若列格式中存在不止一个X项,则这些标为X的列的列宽相同,因此,一般不将内容较短的
列设为X。标为X的列均为左对齐,因此其余列一般选为l(左对齐),这样可使得表格美观
,但也可以选为c或r。

\subsubsection{对物理量符号进行注释的情况}[The condition when physical symbols
need to be annotated]

为使得对公式中物理量符号注释的转行与破折号“———”后第一个字对齐,此处最好采用表格
环境。此表格无任何线条,左对齐,且在破折号处对齐,一共有“式中”二字、物理量符号和
注释三列,表格的总宽度可选为文本宽度,因此应该采用\verb|tabularx|环境。由
\verb|tabularx|环境生成的对公式中物理量符号进行注释的公式如式(\ref{eq:1})所示。
\begin{equation}\label{eq:1}
\ddot{\boldsymbol{\rho}}-\frac{\mu}{R_{t}^{3}}\left(3\mathbf{R_{t}}\frac{\mathbf{R_{t}\rho}}{R_{t}^{2}}-\boldsymbol{\rho}\right)=\mathbf{a}
\end{equation}
\begin{tabularx}{\textwidth}{@{}l@{\quad}r@{———}X@{}}
式中& $\boldsymbol{\rho}$ &追踪飞行器与目标飞行器之间的相对位置矢量;\\
&  $\boldsymbol{\ddot{\rho}}$&追踪飞行器与目标飞行器之间的相对加速度;\\
&  $\mathbf{a}$   &推力所产生的加速度;\\
&  $\mathbf{R_t}$ & 目标飞行器在惯性坐标系中的位置矢量;\\
&  $\omega_{t}$ & 目标飞行器的轨道角速度;\\
&  $\mathbf{g}$ & 重力加速度,$=\frac{\mu}{R_{t}^{3}}\left(
3\mathbf{R_{t}}\frac{\mathbf{R_{t}\rho}}{R_{t}^{2}}-\boldsymbol{\rho}\right)=\omega_{t}^{2}\frac{R_{t}}{p}\left(
3\mathbf{R_{t}}\frac{\mathbf{R_{t}\rho}}{R_{t}^{2}}-\boldsymbol{\rho}\right)$,这里~$p$~是目标飞行器的轨道半通径。
\end{tabularx}\vspace{3.15bp}
由此方法生成的注释内容应紧邻待注释公式并置于其下方,因此不能将代码放入
\verb|table|浮动环境中。但此方法不能实现自动转页接排,可能会在当前页剩余空间不够
时,全部移动到下一页而导致当前页出现很大空白。因此在需要转页处理时,还请您手动将
需要转页的代码放入一个新的\verb|tabularx|环境中,将原来的一个\verb|tabularx|环境
拆分为两个\verb|tabularx|环境。