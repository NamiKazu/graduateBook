\chapter{基于代码审查图的代码架构和质量信息可视化}
\section{引言}


在软件开发过程中,开发者通常会经历多个阶段。在项目的初始阶段,开发者需要阅读和理解已有的代码,这是熟悉软件项目的第一步。然而,对于大型项目而言,由于项目代码量庞大,涉及的模块和功能众多,这一过程通常需要耗费大量的时间和精力。除此之外,开发者在对软件进行修改时,如添加新功能或修复缺陷,通常需要深入了解修改代码的上下文信息。如果对上下文理解不清晰,可能会导致变更不完全或不准确,进而影响软件质量。在软件开发的后期,开发者往往需要作为代码审查者参与到代码审查过程中。代码审查的主要目的是评估变更后的代码是否符合质量标准,是否能够顺利地合并到主分支中。这一过程不仅在协作开发中至关重要,也是确保软件质量的有效手段。然而,代码审查往往需要投入大量的时间和精力\cite{花子涵2024代码审查自动化研究综述}。审查者不仅需要对变更的代码本身进行分析,还需要理解这些代码所处的上下文,才能做出正确的评估。

因此,无论是作为开发者还是审查者,理解软件项目的结构和代码是至关重要的。只有深入掌握软件的整体架构和各模块之间的关系,才能在后续的开发和审查过程中保证代码质量。然而,传统的代码阅读和理解方式不仅需要消耗大量的时间和精力,还难以确保高效性和准确性,尤其是在面对庞大复杂的代码库时。

为了提高代码理解的效率并减少人为错误,本文提出了一种基于代码审查图的代码架构和质量信息可视化的方法。通过将项目中的各个方法和全局变量表示为图的节点,用边表示方法与方法之间、方法与全局变量之间的依赖关系、耦合关系和变更影响关系,从而形成一个结构化的代码关系图。这样的图形化展示方式能够帮助开发者和审查者从宏观的角度掌握整个软件项目的架构和各个模块之间的关系,进而提升对代码的理解效率。通过这种方式,开发者和审查者可以更直观地识别出项目中的关键部分及其相互依赖关系,从而在变更和审查过程中更高效地评估代码的质量和影响。

\section{基于代码中间表示的代码质量度量模型}

\subsection{代码质量度量模型}

代码质量度量模型用于评估和量化软件代码质量,本文研究对象是方法以及模块的代码质量,因此从以下4个方面的质量特性来

代码内聚度和代码耦合性是衡量软件设计质量的两个核心指标,它们直接反映了代
码模块化质量。代码内聚度指的是模块(如文件、类或组件)内部元素之间的相关性。高内聚度意味着模块内的所有元素都紧密地围绕着一个单一的、明确的功能,代码更容易理解和维护\cite{2014Service}。代码的耦合性则描述了模块之间的相互依赖。低耦合度意味着模块之间的依赖关系较小,每个模块都可以独立地执行其功能,而不需要过多地依赖其他模块。低耦合度的代码更容易测试和维护\cite{2013Ahe}。


基于方法摘要和全局变量信息表,我们计算如下代码度量,用于分析代码质量。

\begin{figure}[h]
\centering
\includegraphics[width = 1.0\textwidth]{质量模型.jpg}
\caption{扇入扇出实例}
\end{figure}


\subsection{基于内聚度缺乏度的内聚性分析}

LCOM(Lack of Cohesion in Methods)系列指标是根据模块内聚度的缺乏程度来衡量模块的内聚度的指标。在本文中,
面向对象语言以类为研究范围进行计算内聚度,非面向对象的语言以文件为研究范围进行计算,类中的成员属性对应文件中的全局变量,类中的成员方法对应文件中定义的方法。LCOM 指标的核心思想是度量一个类中方法对实例变量(属性)的共享程度。不同版本的 LCOM 有着不同的计算方法和含义,体现了不同的侧重点。这里一共计算以下四个指标,

(1)LCOM1,含义是不引用相同字段的方法对数目\cite{1994Ametr}。计算公式如式(2-1)。
\begin{equation}
LCOM1 = C_{n}^{2}-e
\end{equation}

其中n 是文件中的方法总数,e 是引用相同字段的方法对。以图2-4为例介绍计算方式,其中椭圆表示方法,点表示变量,点在椭圆内表示该方法引用了该变量。LCOM1值为\(C_{6}^{2} - 5 = 10\)。

\begin{figure}[h]
\centering
\includegraphics[width = 0.7\textwidth]{内聚度示例.jpg}
\caption{示例模块}
\end{figure}
    

(2)LCOM2,含义是不引用相同字段方法对与引用相同字段方法对数之差\cite{1996Coupling}。其计算公式如式(2-2)。

\begin{equation}
    {LCOM2}=\left\{
        \begin{array}
        {c}P-Q,  ifP\geq Q \\
        0,  otherwise
        \end{array}\right.
\end{equation}

其中,P 是不共享实例变量的方法对的数量,Q 是共享实例变量的方法对的数量。
如果 LCOM1 的结果为负数,则被置为 0。图2-4模块中,不共享变量的方法对P为10,共享变量的方法对Q为5,LCOM2值为P-Q=5。

(3)LCOM3是对前两种指标的进一步改进,其计算公式如式(2-3):
\begin{equation}
LCOM3 = \frac{\left( \frac{1}{a} \sum_{j=1}^a \mu(A_j) \right) - m}{1 - m}
\end{equation}

其中\( m\)为文件中的方法数,\( a\)表示文件中的变量数,\( mu(A_j)\)表示的是引用变量\(A_j\)的方法数。如图2-5所示的文件中有3个方法和3个变量,计算方式如图所示。
\begin{figure}[h]
\centering
\includegraphics[width = 0.7\textwidth]{LCOM3.jpg}
\caption{LCOM3计算示例}
\end{figure}



(4)LCOM4,含义是以方法和变量为顶点,方法引用字段或方法之间有调用关系则两节点之间有条边构成图的连通分支数\cite{1995Measuring}。计算时,根据深度优先搜索的方式,计算图中的连通分支数,得到的值即为LCOM4。如图2-6所示的两个文件的LCOM4的值分别为2和1。

\begin{figure}[h]
\centering
\includegraphics[width = 0.7\textwidth]{LCOM4.jpg}
\caption{LCOM4计算示例}
\end{figure}

\subsection{基于连通性的内聚性分析}
TCC(Tight Class Cohesion)和 LCC(Loose Class Cohesion)是用于衡量模块内
聚度的指标,这两个指标主要关注于模块中方法之间的连通关系,核心思想是通过
分析模块中方法如何相互作用以及如何访问共同资源(如全局变量)来评估模块的内聚度。


(1)TCC,含义是有连通关系的方法对数与总方法对数的比值\cite{1995Cohesion}。
TCC 关注于模块中方法之间的“直接连接”。如果两个方法直接共享访问同一个变
量,则认为这两个方法是直接连接的。计算公式如式(2-4)。
\begin{equation}
{TCC} = \frac{e}{C_{n}^{2}}
\end{equation}

其中\(n\)是文件中的方法总数,\(e\)是图中的直接连接边数。

(2)LCC则基于方法间接引用共同字段的关系进行计算\cite{1995Cohesion}。
LCC 除了考虑直接连接的方法对外,还包括了间接连接的方法对。如果两个方法不
是直接连接,但可以通过一系列的方法调用或变量引用来连接,则认为它们是间接连接的。LCC 的值基于模块中直接或间接连接的方法对占所有可能方法对的比例来计算。因此,LCC 的值通常不低于 TCC 的值,并且提供了一个更宽泛的模块内聚度视角。计算公式如式 (2-5):
\begin{equation}
{LCC=\frac{e+e_{indirect}}{C_{n}^{2}}}
\end{equation}

其中\(n\)是文件中的方法总数,\(e\)是图中的直接连接边,\(e_{indirect}\)是除直接连接边的边数。如图2-7是计算LCC和TCC的例子,左图中通过方法AB通过变量x直接连接,方法CD通过变量y直接连接,直接连接和间接连接都是2。而右图中直接连接是AB、BC、BC和CD,间接连接是AD和BD,因此计算结果如图。

\begin{figure}[h]
\centering
\includegraphics[width = 0.7\textwidth]{TCCLCC.jpg}
\caption{TCC和LCC计算示例}
\end{figure}


\subsection{方法间耦合性分析}

耦合是在软件架构中用来描述模块间相互依赖和连接程度的一个重要指标。耦合度的高低直接影响到系统的维护性和可扩展性。在现有的研究和实践中,耦合度通常被细分为六个等级,如表2-1所示,这些等级从高到低反映了模块间依赖的紧密程度。本文关注的是方法与方法之间的耦合性,方法间的耦合性反映了不同方法之间的依赖关系,它直接影响代码的可读性、可测试性以及后续的维护和扩展。通过深入分析方法级别的耦合性,研究方法如何通过参数传递、调用关系、共享全局变量等方式相互依赖,我们可以更准确地识别潜在的设计缺陷和优化机会,从而提高系统的模块化程度,增强系统的可维护性和可扩展性。

\begin{table}[htbp]
\caption{软件架构中耦合性分类}
\vspace{0.5em}\centering\wuhao
\begin{tabular}{ccccc}
\toprule
耦合性类别 & 描述 & 耦合程度 & 本文是否分析 \\
\midrule
内容耦合 & 模块直接访问或修改另一个模块的内部数据 & 6 & 否\\
公共耦合 & 模块访问同一公共数据环境 & 5 & 是 \\
外部耦合 & 模块共享全局简单数据结构 & 4 & 是 \\
控制耦合 & 模块传递控制信息,影响计算流程 & 3 & 否 \\
标记耦合 & 通过参数传递复杂数据结构信息 & 2 & 是 \\
数据耦合 & 通过参数传递简单数据 & 1 & 是 \\
\bottomrule
\end{tabular}
\end{table}


内容耦合是耦合度最高的一种形式,它表示一个模块能够直接访问或修改另一个模块的内部数据和结构。在方法级的耦合分析中,这种耦合形式通常不被考虑,因为方法间的直接数据访问往往通过参数传递或者 API 调用实现,而不是直接的内容访问。

公共耦合发生在多个模块共同访问某个全局数据环境时。这种数据环境可能是全局数据结构、全局变量或内存公共区域等。在提取到的全局变量表中,对于复杂数据结构如结构体和数组,其引用点所在的方法之间均存在公共耦合关系。


外部耦合与公共耦合相似,但区别在于它涉及的是对全局简单变量的访问。例如,当多个模块访问或修改相同的全局简单类型变量时,则这些模块之间存在外部耦合。


控制耦合指模块之间传递信息中包含用于控制模块内部的信息。在提取到的方法摘
要表中,遍历方法,如果该方法调用其他方法时,对应方法的参数列表中有变量决定了被调用方法中的计算流程,则方法之间存在控制耦合关系。由于本文不考虑分析方法内部的控制逻辑,因此不提取此种耦合。


标记耦合指通过参数表传递数据结构信息,调用时传递的是数据结构。在方法摘要
表中提取了方法的参数列表,包括参数名和参数类型,根据参数类型,可以确定
参数表中是否包含复杂类型。除此之外,在方法的调用表中,也提取了方法调用的
其他方法,结合这两个信息,即可确定两个方法是否存在着标记耦合关系。


数据耦合指通过参数表传递简单数据。与标记耦合类似,根据参数类型可以确定参
数是否全部为基本类型,结合方法调用表,即可确定两个方法是否存在数据耦合。
\subsection{方法扇入扇出度量分析}

方法的扇入(Fan-in)和扇出(Fan-out)是软件工程中用于衡量方法复杂性和模块
间依赖关系的两个指标。

\begin{figure}[h]
\centering
\includegraphics[width = 0.8\textwidth]{扇入扇出.png}
\caption{扇入扇出实例}
\end{figure}
    

扇入是指调用某个方法的不同方法的数量。它表示了一个方法对其他方法的依赖程
度。扇入值较高的方法通常被认为是重要的或核心的,因为它们被多个其他方法所依赖。
高扇入值可能意味着该方法执行了一个基础或共享的任务。


扇出是指某个方法直接调用的不同方法的数量。它表示了一个方法对其他方法的影
响程度。扇出值较高的方法可能更复杂,因为它们需要管理和协调更多的方法调用。高
扇出值可能意味着该方法具有较高的责任度,且可能更难以理解和维护。


本文中对于提取到的方法摘要表,遍历每一个方法,统计其调用方法的数量即可计算出该
方法的扇出值,再以该方法名在方法摘要表中搜索调用了该方法的方法,统计总数,得
到的值即为扇入值。

\subsection{基于静态检测工具的代码缺陷检测}

为了更准确地衡量代码质量,本文结合了静态代码分析工具 Cppcheck,对项目中的源代码进行检测和分析。Cppcheck 是一款开源的静态分析工具,专门用于检测 C/C++ 代码中的潜在错误和编码规范问题。它通过静态分析技术,扫描源代码,识别可能存在的内存泄漏、空指针解引用、未初始化变量等常见问题,并提供详细的诊断报告。与传统的编译器警告不同,Cppcheck 不依赖于程序的编译过程,它直接分析源代码的结构和逻辑,从而能够发现更广泛的潜在问题,尤其是那些难以通过常规测试手段捕捉到的错误。

在本研究中,Cppcheck 的检测结果被视为代码质量评估的重要组成部分。Cppcheck会根据问题的严重性和类别将检测出的问题如表2-2所示的六类。

\begin{table}[htbp]
\caption{cppcheck报告问题分类}
\vspace{0.5em}\centering\wuhao
\begin{tabular}{cccc}
\toprule
类别 & 描述 \\
\midrule
error &  严重错误 \\
warning & 潜在的错误或不推荐的做法 \\
style & 代码风格问题,通常与代码格式、命名约定等相关 \\
performance & 性能问题,表明当前方式可能效率不高 \\ 
portability & 平台相关问题,可能导致在不同环境出现不同行为 \\
information & 额外的信息或建议 \\ 
\bottomrule
\end{tabular}
\end{table}

检测结果中包括了诸如错误、警告和代码风格问题等不同级别的信息,这些信息帮助开发者识别代码中的潜在问题并及时修复,进而提高代码的可维护性和健壮性。

Cppcheck 提供的报告与其他度量指标相结合,共同构成了全面的质量评估体系。用户可以根据检测报告对代码质量进行更细致的判断和分析,从而为后续的优化和重构提供科学依据。通过这种方式,本研究实现了对代码质量的多维度综合评价,为开发者提供了更为精确的质量检测和改进方向。



\section{代码审查图}

\subsection{代码审查图构建}

代码审查图主要由两个核心元素构成,即节点和边。其中,节点代表软件项目中的方法或全局变量,边则表示节点之间的各种关系,包括耦合关系、变更影响关系以及依赖调用关系。这些元素的结合能够帮助开发者从全局视角理解和评估项目的结构和质量,尤其是在进行代码审查和代码变更时。

(1)节点属性

在代码审查图中,节点的作用是标识项目中的方法和全局变量。通过前文所述的方法摘要表和全局变量信息表,我们为每个方法和全局变量创建了对应的节点。每个节点都具有多个属性,这些属性能够提供有关节点所代表的方法或变量的关键信息,便于开发者对代码进行全面的审查和分析。

方法属性分为两个主要部分:,首先是方法的基本信息,如表4-1所示。

\begin{table}[htbp]
\caption{代码审查图节点属性-基本信息}
\vspace{0.5em}\centering\wuhao
\begin{tabular}{cccc}
\toprule
    属性 & 描述 \\
\midrule
方法名 & 方法名,由方法所在路径和方法名拼接而成,保证唯一  \\
方法参数 & 方法的参数列表,包括参数的名称和类型   \\
方法内调用方法 & 本方法内调用的其他方法名   \\
方法可作用域 & 表明方法是否全局可用   \\
方法所在模块 &  方法所在模块,目前表示为方法所在文件  \\
模型预测方法模块 & 模型预测的方法应在的模块   \\     
\bottomrule
\end{tabular}
\end{table}

这一部分包括方法的名称、所在模块、方法签名、访问修饰符等基本信息,这些信息有助于开发者快速识别和定位方法的功能和作用。例如,方法的名称可以反映其业务功能,所在模块和调用信息则有助于理解方法的上下文和调用约束。

其次是与代码质量相关的度量和信息,如表4-2所示。

\begin{table}[htbp]
    \caption{代码审查图节点属性-质量相关信息}
    \vspace{0.5em}\centering\wuhao
    \begin{tabular}{ccp{9cm}}
    \toprule
    属性类别 & 属性 & 描述 \\
    \midrule
    \multirow{2}{*}{扇入扇出信息}& 扇入 &  方法的扇入值以及扇入值在项目中的排名比例 \\       
                                & 扇出 &  方法的扇出值以及扇出值在项目中的排名比例 \\   \cline{2-3}
    \multirow{2}{*}{内聚度信息}& LCOM1 &  所在模块的LCOM1值以及相应建议 \\       
                                & LCOM2 &  所在模块的LCOM2值以及相应建议 \\    
                                & LCOM3 &  所在模块的LCOM3值以及相应建议 \\    
                                & LCOM4 &  所在模块的LCOM4值以及相应建议 \\    
                                & TCC &  所在模块的TCC值以及相应建议 \\    
                                & LCC &  所在模块的LCC值以及相应建议 \\   \cline{2-3}             
    \multirow{2}{*}{耦合关系}& 数据耦合 &  与本方法存在数据耦合关系的方法 \\       
                                & 标记耦合 &  与本方法存在标记耦合关系的方法 \\   
                                & 外部耦合 &  与本方法存在外部耦合关系的方法 \\   
                                & 公共耦合 &  与本方法存在公共耦合关系的方法 \\   \cline{2-3}
    变更影响关系 & 变更影响关系 &  与本方法存在变更影响的方法,表明来源为代码克隆、变更历史、模型预测 \\    \cline{2-3}
    缺陷 & 静态代码缺陷 &  由cppcheck检测得到的本方法缺陷 \\      
    \bottomrule
    \end{tabular}
    \end{table}

这一部分基于前文方法提取和统计结果,涵盖了一些与代码质量直接相关的指标,如方法的扇入扇出、内聚度、耦合性等。这些度量和指标的结果将结合项目的具体统计数据或检测结果,向开发者提供有针对性的改进建议。例如,如果某个方法的扇出度位居项目中的前5\%,则可能表明该方法在项目中的依赖关系过于复杂,可能导致高耦合性,进而影响系统的灵活性和可维护性。类似地,如果检测出某方法存在不良耦合,则可能需要开发者重新设计该方法与其他模块的接口,以减少不必要的依赖。

对于全局变量的属性则主要包含表4-3中的信息。主要是对全局变量基本信息的展示,方便开发者快速了解该变量的作用域、使用情况以及与其他代码部分的关联性。通过这些信息,开发者能够更好地理解变量在整个项目中的作用及其潜在的质量风险。

\begin{table}[htbp]
\caption{代码审查图节点属性-全局变量信息}
\vspace{0.5em}\centering\wuhao
\begin{tabular}{cccc}
\toprule
    属性 & 描述 \\
\midrule
变量名 & 全局变量名,由所在路径和变量名拼接而成,保证唯一  \\
变量类型 & 变量类型   \\
被使用方法 & 使用了本全局变量的方法名   \\
变量可使用域 & 表明变量是否全局可用   \\
方法所在模块 &  变量所在模块,目前表示为所在文件  \\  
\bottomrule
\end{tabular}
\end{table}


(2)边的设计

在代码审查图中,边表示节点与节点之间的关系,这些关系揭示了软件系统中各个方法与全局变量之间的相互依赖和影响。根据其性质,边的类型主要分为三类,具体分类如表4-4所示:静态依赖关系、耦合关系和变更影响关系。

其中静态依赖关系分为方法之间的调用关系和方法与全局变量的引用,耦合关系如表4-2中所示共4类,变更影响关系则根据检测方法的不同,设定为3个不同的来源。需要注意的是,由于依赖闭包方法本身是基于依赖图生成的,因此在代码审查图中,这类关系不再单独指出。

每种关系反映了不同层次的代码相互作用,帮助开发者全面理解系统的结构和潜在的质量风险。

\begin{table}[htbp]
\caption{代码审查图边分类}
\vspace{0.5em}\centering\wuhao
\begin{tabular}{cccc}
\toprule
属性 & 描述 \\
\midrule
静态依赖关系 & 含方法之间调用、方法引用全局变量两种  \\
耦合关系 & 含数据耦合,标记耦合,外部耦合,公共耦合四种   \\
变更影响关系 & 含代码克隆、变更历史、模型预测三种  \\
\bottomrule
\end{tabular}
\end{table}



\subsection{代码审查图可视化}

本节从代码审查图的可视化方案和交互方案两个方面展开介绍。

1.可视化方案 

代码审查图的可视化方案基于开源项目 G6。G6 是一个强大的图形可视化引擎,提供了绘制、布局、分析、交互、动画等全方位的图形可视化基础功能,具有简单易用且完备的特性。G6 具有两个显著优势。

(1)数据与可视化图形分离:在使用 G6 时,用户只需将图的数据组织为 JSON 格式,如图4-3所示,包括节点信息和边信息,直接传递给 G6 即可自动生成对应的力导向图。这种数据与图形的分离不仅简化了开发流程,还提高了数据的灵活性和可操作性,便于进行后续的数据更新和图形重绘。

\begin{figure}[h]
\centering
\includegraphics[width = 0.6\textwidth]{G6图数据示例.jpg}
\caption{G6图数据示例}
\end{figure}

(2)高度的定制能力:G6 提供了丰富的图形展示配置选项,用户可以根据需求自由选择不同的样式和布局方式。如果 G6 内置的元素不满足特定需求,它还支持用户自定义节点、边及其他元素,使得图形展示更加贴合实际应用场景。

本文使用G6内置节点和边实现代码审查图的可视化。G6的节点构成共包含6部分,其中label表示文本标签,通常用于展示节点的名称或描述,本文中将节点属性赋值给label,便于用户查看属性相关信息。G6的边的构成共包含4部分,label具有同样的功能,将边的类别用于label。让 G6 加载此数据源进行展示,就实现了同时也实现了计算逻辑与图形可视化的有效分离。


2.交互方案

对于软件项目这样的分析对象,方法和全局变量的数量常达到千级别,这样的级别对于一个图来讲,很难在图中展示完所有的信息,因此需要用户交互,来展示更详细的信息。表4-5展示了目前代码审查图的交互和对应的逻辑设计。


\begin{table}[htbp]
\caption{代码审查图交互和逻辑设计}
\vspace{0.5em}\centering\wuhao
\begin{tabular}{cccc}
\toprule
交互方式 & 业务逻辑 \\
\midrule
视角缩放 & 操作鼠标滚轮对图进行缩放  \\
视角移动 & 鼠标拖拽移动整个代码审查图   \\
聚焦节点 & 光标悬停在节点上显示节点的方法名/变量名  \\
移动节点 & 鼠标长按节点拖拽可移动节点 \\
查看节点属性 & 鼠标点击节点展开节点属性  \\
聚焦关系 & 光标悬停在边上显示边的类型  \\
查看关系信息 & 鼠标点击节点展开节点属性  \\
节点筛选 & 通过点击筛选节点按钮,确认是否筛选掉孤立节点 \\
\bottomrule
\end{tabular}
\end{table}



\subsection{代码审查报告生成}

在软件开发和代码审查的过程中,开发者通常可以借助代码审查图聚焦于代码的上下文,帮助发现局部代码的问题。然而,当软件开发完成,开发者希望从全局角度对软件项目的整体质量进行衡量时,仅依靠代码审查图可能会存在质量信息过于分散、不易聚焦的问题。因此,本研究进一步提出通过生成文档化的代码审查报告,为开发者提供统一的代码质量概览,帮助其全面掌握项目的质量状况。

代码审查报告的核心目标是揭示软件项目中存在的关键质量问题,并以本文提取的代码质量属性为主线,系统性地向用户报告代码中的潜在问题。具体来说,报告主要涵盖以下几个方面的内容:

\begin{itemize}
    \item 内聚度信息:报告中将重点标记内聚度最差的 5\% 模块,并提供相应的统计信息。这些模块通常在逻辑结构上松散、职责分散,可能是代码设计需要改进的关键部分。开发者可通过这些信息快速识别项目中存在高维护风险的模块。
    
    \item 不良的耦合信息:包括同模块内的公共耦合、不同模块的公共耦合以及外部耦合。这些耦合关系可能导致模块之间的高依赖性和低灵活性,影响系统可维护性。
    
    \item 复杂度信息:报告将筛选出扇入扇出指标最差的 5\% 模块,提供详细数据说明。这些模块往往由于过多的依赖关系或调用关系而难以维护,是优化的重点对象。
    
    \item 代码缺陷与规范信息:cppcheck 检测出的代码缺陷和不符合规范的代码信息,并给出相应的建议。
    
    \item 不良变更影响:通过分析代码中的变更影响关系,报告将重点指出项目中存在变更影响关系数量最多的前 5\% 方法。这些方法通常具有较高的变更复杂度,是项目维护中的风险点。
    
    \item 基于大模型的模块预测结果:报告列出那些标签不属于原始模块的方法名称。此类方法可能存在职责划分不当或模块归属不合理的情况,报告将此信息提供给用户,供其判断是否需要重新划分模块归属,从而优化模块结构。
\end{itemize}

为了帮助用户在代码开发过程中防止变更不完全,报告还将列出项目中所有的代码变更影响关系。用户可以参考这些信息,在变更时全面评估影响范围,降低遗漏风险。

通过对这些信息的整合与分析,生成的报告文档为用户提供了一份全面的代码质量概览,既可以帮助用户识别代码中的潜在问题,又能为系统的后续优化与维护提供指导性建议。

\section{实验结果与分析}

\subsection{实验环境与评价方法}

本章依旧使用第二章中的项目为示例项目。

(1)实验环境

实验环境如表4-6所示。

\begin{table}[htbp]
\caption{实验环境}
\vspace{0.5em}\centering\wuhao
\begin{tabular}{cccc}
\toprule
    环境 & 信息 \\
\midrule
操作系统 & macOS Ventura v13.5.2  \\
Intellij pycharm & 2021.1.1   \\
python & 3.7   \\
Java & 1.8   \\
G6 & g6.min.js 4.3.11  \\  
libclang & 15.0.7  \\ 
pycparser & 2.21  \\
\bottomrule
\end{tabular}
\end{table}



(2)评价方法


对于代码审查图生成方法,首先对示例项目生成代码审查图,验证可视化方案的需求,根据代码审查图分析前文中实验结果中的实例研究,以说明代码审查图相较文字的优越之处。

对于代码审查报告的生成,则是验证审查报告的格式是否符合设计方案。

\subsection{实验结果分析}


将实验项目的github仓库克隆到本地,对每个项目进行按前文所述步骤提取代码中间表示,包括抽象语法树、方法摘要表和全局变量信息表,再进一步按2.5节所述方法,提取代码质量度量和缺陷信息。在提取到质量度量后,筛选质量较差的代码分析结果报告给用户,收集用户的反馈,衡量方法的有效性。报告的标准如下:

(1)内聚度

内聚度是衡量一个模块内部各个组件(本文中是方法和全局变量)之间关系的紧密程度的指标。内聚度越高,则意味着模块内的各个部分越紧密,职责明确且功能聚焦。然而,仅凭数值的计算,用户可能难以直观理解模块的内聚度高低。因此本文通过对同一项目中各模块内聚度进行排名,向用户报告内聚度最差的前5\%模块,以帮助用户识别和定位项目中内聚度较差的模块。随着用户不断进行改进,模块的内聚度将逐步提升并收敛,最终整个项目达到平衡的状态。

(2)耦合性

本文中提取了四种不同的耦合类别,每种类别的耦合程度并不相同,数据耦合、标记耦合、外部耦合、公共耦合,耦合程度依次递增。\inlinecite{迟曲2011关于软件设计的模块独立性分析}中提到数据耦合是最佳的耦合类型,而外部耦合和公共耦合则对系统的质量和维护性有负面影响。这是由于数据耦合和标记耦合都通过参数传递数据,方法的输入输出明确,易于测试和维护,而外部耦合和公共耦合共享公共数据,可能会影响程序的可测试性和问题的排查诊断。

本文进一步将代码是否处于同一模块考虑在内,因为对于同一模块内,代码往往有共同的上下文和职责范围,因此可适当对耦合标准放宽。因此对与在同一模块内的方法,报告公共耦合,对于不在同一模块内的方法,报告外部耦合和公共耦合,将这两种情况作为不良耦合报告给用户。

(3)扇入扇出

传统标准中要求尽量高扇入低扇出,高扇入表示其他模块依赖该模块,模块的功能可能是系统中其他模块的核心或基础,复用性较好;低扇出表示该模块没有过多地依赖其他模块,因此它具有更高的独立性,更易维护。同内聚度类似,仅计算数值或统一设置阈值的方式,用户难以直观理解,因此这里也对同一项目中扇入扇出值进行排序,向用户报告最差的前5\%的方法。

(4)静态分析工具

本文结合了静态分析工具Cppcheck对项目代码进行检测。考虑到Cppcheck的严重等级分类能从错误、警告、风格、性能、未定义行为等多个角度提供细致分析,每个类别都关注特定的代码问题,不仅帮助开发者准确识别出可能导致崩溃或异常行为的缺陷,还能够进行规范检查,发现影响程序效率、可读性、资源管理等方面的潜在问题。因此本文将向用户报告表2-2中六种类型的问题,作为质量评估的一部分,提供给用户进行判别。

总的来讲,这四种度量向用户报告的标准如表2-4所示。

\begin{table}[htbp]
\caption{各度量报告标准}
\vspace{0.5em}\centering\wuhao
\begin{tabular}{cccc}
\toprule
度量 & 报告标准 \\
\midrule
内聚度 &  6项指标分别报告内聚度最差的前5\%模块 \\
\multirow{2}{*}{耦合性} &  同模块内报告公共耦合\\
                        & 不同模块报告公共耦合、外部耦合 \\
扇入扇出 & 分别报告最差的前5\%方法 \\
静态工具检测 &  报告6种类型问题\\ 
\bottomrule
\end{tabular}
\end{table}

为了验证质量度量的可靠性与实用性,本文邀请了四位经验丰富的软件开发者,对报告的每一项的准确性和可接受性进行评估。评估的标准包括以下几个方面:

\begin{itemize}
    \item 内聚度差、耦合不良、扇入过低、扇出过高和缺陷的检测结果是否能为用户所接受:评估这些指标的报告内容是否准确反映了代码中的问题,且是否能够为开发者提供有效的指导。
    \item 指标的解释与建议是否有助于代码优化:不仅是检测结果本身,还需要评估报告中对检测结果的解释是否充分。
    \item 报告结果与实际代码质量的关联性:通过比较报告中的分析结果与开发者实际编码过程中所遇到的问题,验证报告的实用性。
\end{itemize}

对于报告的每一项,如果用户认为符合标准,则认为用户接受对应的结果,则将对应的项反馈结果标记为1,统计用户的接受率,接受率指标计算公式如式(2-6)
\begin{equation}
    AR_n =  \frac{\frac{1}{N} \sum_{i=1}^{N} A_i}{T} \times 100\%
    \end{equation}
其中AR表示接受率,A表示开发者接受的项数,T表示报告的总项数,N表示软件开发者数。

1. 内聚度实验结果及实例分析

(1)内聚度实验结果

表2-5展示了对于内聚度报告项用户的接受率,从实验结果可以看出,该方法表现良好,可靠性较高。具体来讲,指标基本都在70\%以上。通过进一步对比,发现在TheAlgorithms、antiword等比较简单的项目中,内聚度在6项指标中常出现相同的接受率,经过分析发现这是因为在同一项目中,尽管6项指标的计算方式和侧重点都不同,但是最差的前5\%基本上是重合的,这意味着一个模块如果在LCOM1中表现的很差,在其他指标中也通常表现得很差,这符合通常认知,同时也侧面验证了指标的准确性。但是对于FFmpegKit等较复杂的项目,6项指标则不太相同,体现了不同内聚度计算方式的侧重点。

\begin{table}[htbp]
\caption{内聚度接受率}
\vspace{0.5em}\centering\wuhao
\begin{tabular}{ccccccc}
\toprule
项目名称 & LCOM1 & LCOM2 & LCOM3 & LCOM4 & TCC & LCC \\
\midrule
TheAlgorithms & 89.2 & 74.9 & 74.9 & 89.2 & 74.9 & 74.9 \\
antiword-0.37  & 91.3 & 84.2 & 84.2 & 91.3 & 84.2 & 84.2 \\
jemalloc-5.3.0 & 82.2 & 82.2 & 82.2 & 82.2 & 88.6 & 88.6 \\
libbpf-1.1 & 87.3 & 83.4 & 87.3 & 87.3 & 93.7 &  93.7 \\
librdkafka-2.1.0 & 73.7 & 73.7 & 73.7 & 73.7 & 78.4 & 78.4 \\
FFmpegKit-5.1.0 & 71.3 & 69.2 & 74.9 & 76.6 & 89.3 & 84.9 \\

\bottomrule
\end{tabular}
\end{table}


(2) 内聚度实例分析

对内聚度实验结果进行进一步分析,这里选取 antiword 项目中的内聚度表现最差的模块 misc.c 文件模块进行分析。为了更好地理解这一现象,进一步检查了该模块的源代码,并对其结构进行了统计分析。具体来看,misc.c 文件中一共定义了 1 个全局变量和 15 个方法,其中 11 个方法并未被同一模块内的其他方法调用,只有 3 个方法之间存在相互调用关系,而仅有 2 个方法使用了该模块定义的全局变量。

通过这个结果可以发现,尽管该模块包含了一定数量的方法和变量,但模块内各个方法和变量之间的依赖关系极为薄弱。大部分方法相互之间没有调用关系,且仅有少数方法与全局变量发生交互。这样的结构表明,模块内的功能划分较为松散,各个功能单元之间缺乏必要的协作和紧密联系。因此,该模块的内聚度较差,功能的集中性和一致性较低,导致其内聚度值显著较低。因此,通过对 misc.c 文件内聚度的分析,我们不仅可以得出该模块的内聚度较差的结论,还能为未来的重构和优化提供指导意见。例如,增强模块内部方法之间的调用关系,优化变量的使用方式,以提高模块的内聚度,从而提升系统的整体质量和可维护性。

2. 其他度量实验结果及实例分析

表2-6展示了对于其他度量报告项用户的接受率,从实验结果可以看出,不同项目在各项度量指标上的接受率存在一定差异。大部分项目的耦合性接受率较高,而扇入扇出的接受率普遍较低,大部分的项目上只达到了40-50\%的水平,而所有项目对静态工具检测出来的缺陷上的接受率都较高,说明静态工具的可靠性较好。接下来对所有指标的实验结果进行进一步分析。

\begin{table}[htbp]
\caption{其他度量接受率实验结果}
\vspace{0.5em}\centering\wuhao
\begin{tabular}{cccc}
\toprule
项目名称 & 耦合性接受率 & 扇入扇出接受率 & 静态工具缺陷接受率 \\
\midrule
TheAlgorithms & 99.2 & 87.6 & 100.0 \\
antiword-0.37 & 84.2 & 56.4 & 98.0 \\
jemalloc-5.3.0 & 77.8 & 66.3 & 100.0 \\
libbpf-1.1 & 68.3 & 42.1 & 100.0 \\
librdkafka-2.1.0 & 63.7 & 48.7 & 98.8 \\
FFmpegKit-5.1.0 & 54.0 & 43.3 & 94.6 \\
\bottomrule
\end{tabular}
\end{table}

(1)耦合性结果分析和实例分析

耦合性的接受率在几个项目中的表现不尽相同。对于结构较为简单的项目,如antiword和jemalloc,接受率均能达到较高的水平,在TheAlgorithms项目上甚至能达到99\%以上的接受率。而对于更大型更复杂的项目,接受率则略微降低。

进一步分析如此表现的原因,对于TheAlgorithms,这个项目是对各种算法的开源实现,模块和模块之间几乎不存在相互依赖和调用的情况,每种算法独立开发,实际上是库函数,仅在需要时供开发者调用。总的来讲,简单项目结构较为简单,各个模块功能明确,职责聚焦,所以耦合性不良的样例很少,共检测出15例。维护起来也更加容易,用户更容易接受。

而在复杂项目中,如FFmpegKit,耦合性接受率相较其他项目低一些。这里以FFmpegKit项目中不被用户所接受的样例进行分析。在这个样例中,方法probe\_file与另外三个模块的方法发生了公共耦合,具体来讲这几个方法共同使用了fftools\_ffprobe.c文件中定义的全局变量nb\_streams,这是一个整数。而不被用户所接受的理由是,这个全局变量实际上是一个共享全局状态,它代表了流的数量,需要在不同平台上保持一致,如表中所示,需要在android、linux、apple的环境中都保持一致,才能保证程序的正确性和一致性。在这种情况下,多个方法访问并操作该全局变量是合理的,因为这种做法确保了平台间的一致性。

\begin{table}[htbp]
\caption{FFmpegKit项目中不被用户接受的不良耦合实例}
\vspace{0.5em}\centering\wuhao
\begin{tabular}{cp{12cm}}
\toprule
共同访问变量 & 方法名 \\
\midrule
\multirow{4}{*}{nb\_streams} & ffmpeg-kit/apple/src/fftools\_ffprobe.c@probe\_file \\
                              & ffmpeg-kit/android/ffmpeg-kit-android-lib/cpp/fftools\_thread\_queue.c@tq\_alloc  \\
                              & ffmpeg-kit/linux/src/fftools\_thread\_queue.c@tq\_alloc  \\
                              & ffmpeg-kit/apple/src/fftools\_thread\_queue.c@tq\_alloc  \\
\bottomrule
\end{tabular}
\end{table}

通过对这个样例的分析可以看出,对与大型、复杂的软件项目来讲,各模块和方法之间的依赖关系可能已经非常复杂,需要大量时间来理解现有结构并分析方法间的相互影响,对于某些功能,也的确缺乏耦合性更低的实现方式,所以接受率更低。

(2)扇入扇出接受率分析

对于扇入和扇出值指标,观察到该指标的接受率较低。通过实例分析发现,大多数项目的开发模式并不完全符合高扇入、低扇出的设计原则。

特别是在低扇入的报告项中,接受率尤为偏低。实际分析表明,许多方法仅被单次调用,虽然这些方法的复用性较差,但它们更为独立,且较少依赖于其他模块的实现。这种独立性使得这些方法更容易进行单独测试、维护和扩展。考虑到软件的长期发展,这些低扇入的方法未来仍有可能被其他模块调用。因此,低扇入方法在某些情况下是有其优势的,用户难以直接接受此类报告项,因此拉低了整体的接受率。

而对于高扇出的报告项,大部分是可以接受的,但是也存在某些高扇出的方法是合理的,比如有些高扇出方法实际上是中心化的模块。

\begin{figure}[h]
\centering
\includegraphics[width = 0.5\textwidth]{扇出高例子.jpg}
\caption{vGetPropertyInfo方法结构}
\end{figure}
    

这里以antiword项目中properties.c文件的vGetPropertyInfo方法为例,该方法主要根据传入的Word版本,从文档中提取各种属性信息。该方法的扇出度为30,是项目中扇出度较大的一个,方法可能复杂度过高,因此被报告属于不良的扇出,但这一报告并未得到用户的接受。其原因在于,该方法的核心结构是一个包含多达9种不同情况的 switch 语句,每种情况又细分为多个子情况,因此导致总共产生了30个方法调用。该方法的合理性在于通过清单式方式集中处理所有功能逻辑,使得用户能够一目了然。此设计不仅减少了冗余代码,避免了方法调用链过长带来的可读性问题,还降低了因方法调用所带来的额外开销,从而优化了整体系统的效率。因此,尽管该方法具有较高的扇出度,其设计思路仍然是有效的,符合高效代码的要求。

(3)静态工具检测接受率分析

静态工具检验得到的代码缺陷的接受率是最高的,经过分析发现只有个别错误,如平台相关问题,与特定编译器相关的警告等不容易被开发者所接受,因为这并不是代码本身的质量问题。


经过分析可以发现,这些度量指标有一定的有效性,但是这些度量聚焦的往往是范围较小的质量问题。如模块内、方法内甚至语句级的质量问题。面对更大规模的软件代码时,无法反映其在架构上的缺陷,难以帮助开发者从宏观的角度上指导代码维护。


2. 代码审查图

(1)代码审查图概览

图 4-4 展示了 Antiword 项目和 TheAlgorithms 项目的代码审查图,直观地反映了两者的结构特征和模块化差异。图中,圆形节点表示方法,方形节点表示全局变量,不同颜色的边则代表了代码元素之间的不同关系:蓝色边表示依赖关系,绿色边表示耦合关系,红色边表示代码变更影响关系。上方的图是未区分模块的全局视图,所有的节点和边混杂在一起,仅从整体上体现了项目的结构复杂度。而下方的图对模块进行了区分,采用颜色区分不同模块,将属于同一模块的节点用相同颜色进行标注,从而进一步突出模块之间的边界和逻辑关系。

\begin{figure}[!h]
    \setlength{\subfigcapskip}{-1bp}
    \centering
    \begin{minipage}{\textwidth}
    \centering
    \subfigure[antiword-未划分模块]{\includegraphics[width=0.4\textwidth]{antiword审查图未上色.jpg}} % 保留中文标题
    \hspace{2em}
    \subfigure[TheAlgorithms-未划分模块]{\includegraphics[width=0.4\textwidth]{aigri代码审查图-未上色.jpg}} % 保留中文标题
    \end{minipage}
    \centering
    \begin{minipage}{\textwidth}
    \centering
    \subfigure[antiword-划分模块]{\includegraphics[width=0.4\textwidth]{antiword代码审查图.jpg}} % 保留中文标题
    \hspace{2em}
    \subfigure[TheAlgorithms-划分模块]{\includegraphics[width=0.4\textwidth]{aigri代码审查图.jpg}} % 保留中文标题
    \end{minipage}
    \vspace{0.2em}
    \caption{代码审查图} % 只保留中文标题
\end{figure}


从图中可以明显看出两个项目在结构特征上的显著差异。Antiword 项目整体模块之间高度协作,共同实现一个完整的功能,因此模块之间的联系较为紧密,表现出较强的耦合性。从可视化图上观察,按模块划分后,可以清晰地看到具有相同功能的模块形成了较为紧密的聚集。同一颜色的节点集中分布,进一步体现了模块的内聚性较高以及逻辑结构的清晰性。

相比之下,TheAlgorithms 项目由于其作为算法库的特性,方法之间的耦合性较低,各模块间的联系相对较弱。从图上来看,不同模块呈现出较为分散的分布,模块内部的聚集程度也较低。整体结构表现为由若干独立模块组成,松散而分离,符合库函数式项目的典型特征。

\begin{figure}[h]
\centering
\includegraphics[width = 1\textwidth]{点击节点.png}
\caption{点击节点展开节点信息}
\end{figure}

此外,用户可以通过点击图中的某个节点来查看该节点的详细信息,包括方法或全局变量的具体描述。这一功能能够帮助用户快速定位关键信息,深入了解特定方法或变量的功能和用途,从而更高效地进行代码审查和理解。



(2)子图展示

代码审查图蕴含了丰富且有价值的信息,通过聚焦于代码审查图中的子图,开发者和审查者可以快速直观地理解代码结构之间的关系。子图清晰地展示了代码结构之间的依赖、耦合以及变更影响关系,使得复杂的代码逻辑更加可视化。

以 antiword 项目中的子图为例,本文聚焦于 vDecodeDIB 方法的代码开发和审查场景。若按照传统方法对该方法进行开发或分析,仅通过人工阅读代码,开发者需要深入解析与其相关的依赖关系和变更影响关系,这可能涉及 300 多行代码的逻辑才能全面掌握方法的上下文。然而,通过代码审查图,开发者只需关注 8 个节点和 9 条边,即可快速获取关键信息。这种直观的可视化显著降低了代码理解的复杂度和成本。

\begin{figure}[h]
\centering
\includegraphics[width = 1\textwidth]{代码审查图子图.jpg}
\caption{vDecodeDIB方法上下文}
\end{figure}

尤其值得注意的是,变更影响关系中包含了两个基于克隆代码的关系。这类关系通常隐匿于代码中,开发者仅靠手动查看代码很难准确发现并记录。代码审查图将这些隐藏的克隆关系显式标出,使开发者在代码修改时能够快速识别潜在的影响范围,从而避免遗漏变更的风险。

对开发者而言,代码审查图能够提供更加直观的视角,帮助其快速掌握代码之间的各种依赖关系和耦合情况。在代码开发过程中,审查图不仅有助于理清模块间的关系,还能有效避免变更不完全的问题。而对于代码审查者,审查图能够快速展示代码的上下文结构,简化复杂逻辑的理解过程,同时高效地识别代码中的变更缺陷。

(3)代码质量属性体现

通过代码审查图,不仅可以直观展示代码的依赖关系和变更影响关系,还能有效地体现代码质量的特征,从而辅助用户理解代码质量较差或较好的具体原因。在第 2.6.2 节中详细解释了 misc.c 文件内聚度表现最差的原因,并统计了该文件内部方法和变量的相关信息。然而,仅依靠文字说明可能难以全面传达问题的全貌,而配合代码审查图,如图4-7所示,可以更加直观地展示该模块的结构问题。

\begin{figure}[h]
\centering
\includegraphics[width = 0.7\textwidth]{内聚度差例子.jpg}
\caption{misc.c 文件对应的代码审查图}
\end{figure}

从图中可以清晰地看出,misc.c 文件的结构非常松散,其内部存在大量孤立节点,缺乏足够的联系和协调关系。这些孤立节点既无法与其他节点形成有意义的逻辑关系,也难以体现出模块内部的高内聚性特征。这种松散的结构正是导致其内聚度表现最差的主要原因。

除此之外还有一些内聚度差和好的模块在代码审查图中可以清晰地展现,如图4-8所示。图中,左侧展示的是内聚度较差的模块,而右侧则展示了内聚度较好的模块。单纯从代码本身出发,往往难以直观地识别出导致内聚度差或较好的具体原因。然而,通过代码审查图,问题的根源得以清晰呈现。具体来说,左侧内聚度差的模块之所以表现为低内聚度,主要是因为存在过多的链式调用,这种设计使得模块之间的依赖关系过于松散,缺乏足够的内在联系。而右侧内聚度良好的模块,则表现出较强的模块化特征,模块内部功能紧密相关,依赖关系清晰且有序,从而促进了代码的高内聚性。通过这种可视化的方式,代码审查图不仅帮助开发者迅速识别出模块内聚度的优劣,还能够为后续的重构与优化提供建议,如左侧内聚度差的模块,可通过合并方法,减少方法数和调用量,达到提高内聚度的效果。

\begin{figure}[!h]
    \setlength{\subfigcapskip}{-1bp}
    \centering
    \begin{minipage}{\textwidth}
    \centering
    \subfigure[内聚度差——存在链式调用]{\includegraphics[width=0.4\textwidth]{存在链式调用.jpg}} % 保留中文标题
    \hspace{2em}
    \subfigure[内聚度良好——关联紧密]{\includegraphics[width=0.4\textwidth]{内聚度好.jpg}} % 保留中文标题
    \end{minipage}
    \vspace{0.2em}
    \caption{内聚度良好和较差的例子} % 只保留中文标题
\end{figure}


在2.6.2节中对高扇出方法的讨论中,高扇出通常意味着方法存在过高的复杂性,这种特征在代码审查图中表现为具有大量的出边的中心化节点。如图4-9所示。图中的中心节点通过大量出边与其他模块或方法产生关联,这种结构不仅揭示了方法的复杂性,也反映了其可能导致的维护难度和潜在的设计缺陷。用户可以通过代码审查图迅速定位到具有高扇出的具体方法,深入了解其上下文信息,进而判断其是否存在不合理的复杂度。当发现高扇出的原因主要源于过多的依赖关系时,开发者可以及时采取相应的优化措施,如重构该方法、简化其职责或调整其与其他模块的关系,从而有效提升系统的可维护性和可扩展性。

\begin{figure}[h]
\centering
\includegraphics[width = 0.4\textwidth]{扇出度最高.jpg}
\caption{扇出度较高的节点}
\end{figure}

对于某些方法,其功能可能已经与当前所在模块的职责不再完全匹配,而与其他模块的结合更加紧密。这一问题在代码审查图中得以有效呈现。得益于力导向图的可视化特性,图中会显示出方法之间的依赖关系与耦合关系,紧密依赖的节点往往聚集在一起。由于同一模块内的节点通常会被赋予相同的颜色,若模块划分不合理,则会出现如图4-10所示的情况。在图中,聚集的节点颜色并不统一,某些节点(如图中的绿色和蓝色节点)与模块的主色调(橙色)不符。这表明,尽管这些方法本应属于其他模块,但它们与当前模块(即橙色模块)之间的耦合关系更为紧密。通过这种可视化,用户能够明确识别出不合理的模块划分,并采取相应的重构措施,例如将这些方法迁移至更合适的模块中,避免产生不良的耦合关系。这种调整不仅能够优化模块的功能划分,还能提高系统的可维护性与可扩展性,减少未来修改时的复杂度。


\begin{figure}[h]
\centering
\includegraphics[width = 0.4\textwidth]{模块分错.jpg}
\caption{模块划分不合适的方法}
\end{figure}


在代码审查图中,我们也能看到一些方法之间的不良变更影响关系。除了直接反映变更影响较多的方法外,代码审查图还能够揭示出由于变更影响分析所导致的模块间不良变更传播问题。如图4-11。该子图涉及三个模块,每个模块的内聚度均较为良好,然而每个模块内都有一个方法与其他两个模块之间存在变更影响关系。这导致当这几个方法变动时,可能影响到不属于同一模块的代码也要跟着改变。更为严重的是,随着变更的涟漪效应,这些变更会进一步扩散,影响到更多模块,造成不必要的连锁反应。这种现象揭示了该系统架构存在潜在的问题——尽管各模块内部结构较为合理,但模块之间的依赖关系过于复杂且紧密,增加了维护和变更时的复杂度和风险。为解决这一问题,优化的建议是尽量减少或消除这种不良的变更影响关系。具体而言,可以通过提取和复用模块间的共同代码逻辑,消除不必要的代码克隆,进而降低跨模块变更传播的可能性。


\begin{figure}[h]
\centering
\includegraphics[width = 0.4\textwidth]{变更影响关系审查图.jpg}
\caption{不良的变更影响关系}
\end{figure}


总而言之,通过代码审查图,不仅能够帮助开发者快速定位模块质量问题,还能直观地理解问题的根源,为后续的优化和改进提供清晰的建议。这种结合文字与图形的分析方式,显著提升了代码质量评估的直观性和说服力。

3. 代码质量评估报告

下图展示了一个实际的代码评估报告示例。由于报告内容较长,这里仅展示了其中的一部分信息。通过代码评估报告,用户能够以结构化的清单形式全面了解软件项目中存在的各类质量问题及其具体位置。这些报告不仅清晰地列出了每个问题的详细描述,还提供了针对性优化或重构的建议。通过这种方式,开发团队可以快速识别代码中的潜在缺陷或性能瓶颈,并根据报告中提供的指导意见,采取有效的措施进行改进。报告的可视化和条目化呈现,帮助用户直观地理解各项问题的优先级和重要性,从而优化项目的维护流程,并提高软件系统的整体质量。

\clearpage

\begin{figure}[h]
\centering
\includegraphics[width = 1.0\textwidth]{质量评估报告.jpg}
\caption{antiword项目代码质量评估报告节选}
\end{figure}


\section{代码审查图的实际应用}

在实际应用中,代码审查图可以有以下三种用法。

\paragraph{辅助开发者的代码维护} 当用户对软件代码进行开发时,可将复杂庞大的项目先通过系统进行分析,得到代码审查图。

\begin{figure}[h]
\centering
\includegraphics[width = 1.0\textwidth]{应用展示审查图.jpg}
\caption{导入项目生成代码审查图}
\end{figure}

对于开发任务所在的代码上下文,通过搜索方法名找到其在代码审查图中的位置,点击节点查看方法详细信息和质量情况,用户可根据相应的建议进行修改,优化代码质量。

\begin{figure}[h]
\centering
\includegraphics[width = 1.0\textwidth]{应用查看节点信息.jpg}
\caption{定位开发任务涉及的方法,查看质量信息}
\end{figure}

根据代码审查图的边,则可以了解当前代码与其他部分的依赖关系、耦合关系和变更影响关系。尤其是变更影响关系,可以帮助用户在进行变更时,提示其依赖型和逻辑型影响的范围,并根据给出的建议,帮助用户安全变更,解决了用户面对复杂软件难以理解、不敢变更、容易变更不完全的痛点。

\begin{figure}[h]
\centering
\includegraphics[width = 1.0\textwidth]{应用边信息.jpg}
\caption{查看方法对应边,按建议安全变更}
\end{figure}

\paragraph{作为审查工具帮助审查者进行审查} 当面对审查任务时,审查者可通过搜索代码审查图定位到当前提交所涉及的节点,通过图中的关系迅速了解代码上下文间的调用以及变更影响关系,通过系统的提示,对开发者的变更进行审查,检查其功能逻辑上是否安全变更,检查其变更操作给软件带来的质量影响是恶化还是优化,从而给出对应的审查结果。

\paragraph{作为质量评估工具对软件代码进行整体的评估} 用户可根据代码审查图,观察软件模块划分,也可导出代码质量评估报告,以清单的方式了解项目代码的质量情况。



\section{本章小结}

本章介绍了基于标签生成的方法模块分类,该方法通过大语言模型生成方法的实际标签进行分类,表明模块划分不正确的方法,通过实验证明,该方法具有一定的参考意义。其次介绍了代码审查图,将软件分析结果以图的形式展示给用户,方便用户从宏观的角度了解项目结构,最后介绍了代码质量报告的生成,通过清单式的质量信息报告,方便用户查询和优化。
